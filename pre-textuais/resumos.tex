%
% ********** Resumo
%

% Usa-se \chapter*, e não \chapter, porque este "capítulo" não deve
% ser numerado.
% Na maioria das vezes, ao invés dos comandos LaTeX \chapter e \chapter*,
% deve-se usar as nossas versões definidas no arquivo comandos.tex,
% \mychapter e \mychapterast. Isto porque os comandos LaTeX têm um erro
% que faz com que eles sempre coloquem o número da página no rodapé na
% primeira página do capítulo, mesmo que o estilo que estejamos usando
% para numeração seja outro.
\mychapterast{Resumo}

Atualmente a Internet das Coisas (Internet of Things - IoT) é uma realidade e suas aplicações têm sido aplicadas em diferentes cenários, cada um com suas próprias restrições e requisitos. Cada nova aplicação IoT desenvolvida para um determinado cenário passa por um processo de validação e homologação antes de ser embarcada em dispositivos reais para produção. Durante o processo de validação dessas aplicações são avaliados aspectos de software, hardware e comunicação. Nesse trabalho será enfatizado aspectos de comunicação, onde dispositivos IoT majoritariamente adotam tecnologias sem fio. Nesse sentido, existem basicamente duas metodologias básicas de validação: testes de bancada reais ou simuladores de redes. Teste de bancada é o método que proporciona resultados mais acurados haja vista utilizar dispositivos reais. \\
Contudo, em cenários mais complexos, com alta densidade de dispositivos ou mobilidade, a validação utilizando testes de banca é bastante custosa. Por outro lado, ambientes de simulação de redes não possuem essas restrições, sendo limitados pela qualidade dos modelos matemáticos que regem o comportamento das simulações, pelos recursos computacionais disponíveis e pelo próprio simulador. Dentre os vários desafios existentes na simulação de redes, incorporar técnicas que utilizem os códigos reais dos protocolos de comunicação apresentam resultados mais próximos dos experimentos de testes de bancada. Uma solução em potencial é a simulação virtual de redes, no contexto desse trabalho, redes virtuais sem fio. \\
A virtualização das redes sem-fio possibilita que variáveis de ambiente possam ser manipuladas e estendidas a um custo muito baixo comparado aos testes de bancada. Portanto, este trabalho consiste em avaliar o desempenho e aplicabilidade de um simulador de redes sem-fio virtuais na validação de novas aplicações de IoT com restrições complexas.



\vspace{1.5ex}

{\bf Palavras-chave}: Análise de Desempenho, IoT, Simuladores de Rede, Virtualização.

%
% ********** Abstract
%
\mychapterast{Abstract}


Nowadays Internet of Things (IoT) is a reality and its applications has been applied in many different scenarios, each with its own constraints and requirements. Each new IoT application developed to a specific scenario is tested and validated before being shipped on real devices for production. During the application’s validation process aspects of software, hardware and communications are evaluated. In this work we will cover the aspects of communications, since IoT devices mostly adopt wireless technologies. In this way, exists basically two validation methodologies: real testbeds or network simulators. Real testbeds is the method that provides the most accurate results as it uses real devices. \\
However, in more complex scenarios, with high density of devices or mobility, the validation process using real testbeds is very costly. On the other hand, network simulation environments doesn’t have these constraints, being limited by the quality of mathematical models used during simulations, the resources available and by the simulator. Among the existing challenges in network simulations, incorporate techniques using real communication protocol’s code produce more accurate results compared to real testbeds. A potential solution is the virtual network simulation, in the context of this work, virtual wireless networks. 
\\
The virtualization of wireless networks enable environment variables being manipulated and extended by a very low cost compared to real testbed methods. Thus, this work aims the performance validation and applicability of a virtual wireless network simulator in validation of new IoT applications with complex constraints.

\vspace{1.5ex}

{\bf Keywords}: IoT, Network Simulators, Performance Analysis, Virtualization.
