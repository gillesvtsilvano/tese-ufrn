%%
%% Capítulo 5: Conclusões
%%

\mychapter{Conclusão}
\label{Cap:Conclusao}

O capítulo final depende do tipo de documento. Nas propostas de tema
deve ser apresentado de forma clara e sucinta o assunto a ser
desenvolvido e o cronograma de execução do trabalho. Nas teses e
dissertações devem ser ressaltadas as principais contribuições do
trabalho e as suas limitações.

As contribuições devem evitar as adjetivações e julgamentos de valor.
Quanto às limitações, não tenha medo de as apresentar: é muito mais
reconhecido um autor que apresenta os casos em que sua proposta não se
aplica do que outro que parece não ter consciência deles.

Escreva (com outras palavras) o que foi realizado e como foi realizado, o que o trabalho descrito no artigo conseguiu melhorar e qual a sua relevância, e quais são as vantagens e limitações das propostas que a tese/dissertação apresenta. Apresente também eventuais aplicações dos resultados obtidos (ou da metodologia, técnica, produto) e ideias de trabalhos futuro que possam melhorar o seu (não apenas apresente, mas indique como pode ser feito).

%\section{Encadernação}
%
%As propostas de tema e as versões iniciais das teses e dissertações
%são impressas em lado único da folha e em espaçamento um e meio. Para
%a encadernação, usa-se geralmente um método simples, tal como espiral
%na lateral das folhas e capa plástica transparente. O número de cópias
%é igual ao número de membros da banca e pelo menos mais uma (para o
%aluno).
%
%As versões finais das teses e dissertações são impressas em frente e
%verso e em espaçamento simples. O número mínimo de cópias é o seguinte:
%\begin{itemize}
%\item 3 cópias para o PPgEEC e a UFRN.
%\item 1 cópia para cada examinador externo que participou da banca.
%\item ao menos 1 cópia para o aluno (não obrigatória).
%\item 1 cópia para o orientador (por cortesia, não obrigatória)
%\end{itemize}
%
%Para a encadernação, deve-se adotar uma capa rígida de cor azul para
%as dissertações de mestrado e de cor preta para as teses de doutorado,
%ambas com letras douradas. Na capa deve constar o título do
%trabalho, o autor e o ano da defesa. Se possível, a mesma informação
%deve ser repetida na lombada do livro.
%
%Para as versões finais, também se exige uma cópia eletrônica (formato
%PDF) do texto, bem como outros dados. Maiores informações podem ser
%obtidas na página do PPgEEC: \url{http://www.ppgeec.ufrn.br/}

\section{Etapas de Homologação do Título}

Depois de defendido, o seu trabalho passará por um processo de homologação. Atualmente, este pode ser feito e acompanhado via SiGAA, menu Ensino $\rightarrow$ Produções Acadêmicas $\rightarrow$ Acompanhar Procedimentos após Defesa.
\begin{enumerate}
	\item A consolidação da atividade de defesa;
	\item A submissão da versão final corrigida da dissertação. Nesta etapa, o seu orientador irá verificar se as falhas apontadas pela banca foram corrigidas. Caso isto tenha acontecido, a sua dissertação/tese será considerada aprovada, passando para a próxima etapa. Caso contrário, você terá que submeter uma nova versão final, para corrigir os erros apontados pelo seu orientador;
	\item A submissão da versão final com a ficha catalográfica, cujas informações podem ser obtidas na biblioteca central, ou solicitadas pelo SiGAA, no \textit{link} ``Solicitar Ficha Catalográfica''. Esta também será avaliada pelo seu orientador. Caso seja aprovada, virá o próximo passo;
	\item A assinatura do termo de autorização da publicação, que pode ser feita via SiGAA, menu Ensino $\rightarrow$ Termo de Autorização;
	\item O envio da versão final para a avaliação da coordenação. Neste ponto será avaliado se o sua monografia/tese satisfaz os requisitos estabelecidos pelo programa. Aqui, \textbf{é muito importante que todas as considerações fornecidas neste modelo seja seguidas rigorosamente}, caso contrário, o seu trabalho será reenviado para fazer as correções necessárias. Caso esteja conforme o exigido pela coordenação, este será aprovado;
	\item A solicitação de homologação do diploma, que pode ser ser acompanhada na Pro-Reitoria de Pós-Graduação.
\end{enumerate}

\section{Para saber mais}

Procure no Google, ora! Brincadeiras a parte, existem inúmeros
tutoriais sobre \LaTeX\ na rede que podem dar maiores informações
sobre o aplicativo. Para conhecer os pacotes disponíveis, uma opção é
o livro \emph{The \LaTeX\ Companion} \cite{LATEX04}, popularmente
conhecido como o ``livro do cachorro''. Outras informações sobre
redação técnica e normas para confecção de teses e dissertações podem
ser encontradas em livros de Metodologia Científica.
